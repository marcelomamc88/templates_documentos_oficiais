Presente em todos os setores, a TI é elemento essencial para o planejamento, execução, controle e avaliação dos processos organizacionais. Isto a torna um elemento extremamente importante para garantir a qualidade dos produtos e serviços gerados.

O avanço tecnológico, porém, tem um alto preço. Equipamentos tornam-se obsoletos com pouco tempo de uso, necessitando atualizações planejadas e contínuas. Da mesma forma, os softwares necessitam ser continuamente atualizados para utilização em novas plataformas e perdem as funcionalidades com a rapidez das mudanças introduzidas pelas leis, regulamentos e mudanças de cenário, necessitando manutenção adaptativa em intervalos cada vez menores.

Neste contexto, cabe ao Sistema de Tecnologia da Informação do Comando da Aeronáutica (STI), entre outros desafios, prover os mecanismos para que o gerenciamento do Ciclo de Vida de cada Sistema de TI necessário ao atendimento das necessidades da Força seja conduzido de forma eficiente e eficaz.

A atuação gerencial, de forma estruturada, será um poderoso instrumento, pois, proverá os mecanismos de planejamento incremental, de controle da execução, de identificação dos desvios e de avaliação dos resultados ao longo de todas as fases do Ciclo de Vida de um Sistema de TI.

Desta forma, a presente Norma foi elaborada somando a experiência adquirida pela Força no processo de gerenciamento do Ciclo de Vida de Sistemas e Materiais aos mais modernos conceitos de gerenciamento de projetos.

Não obstante, nunca será dispensada a participação ativa de todos os Órgãos envolvidos no Ciclo de Vida dos Sistemas de TI, seja no cumprimento do processo de gerenciamento instituído por esta Norma ou na capacitação contínua de seus efetivos nas técnicas de levantamento de requisitos, especificação, desenvolvimento e manutenção de Sistemas de TI, gerenciamento de projetos e garantia da qualidade.

Desta forma, existirá, de fato, um Sistema de Tecnologia da Informação que poderá disponibilizar ferramentas que atendam às necessidades da Força, com custos e prazos de desenvolvimento realistas e, ainda, com garantia de qualidade e de continuidade de operação ao longo do tempo.