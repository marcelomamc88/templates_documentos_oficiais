% \pagestyle{fancy}

\begin{figure}[H]
    \centering
    \includegraphics[scale=0.5]{figuras/brasao_brasil.png}
\end{figure}

\begin{center}
    \textbf{\MakeUppercase{\imprimirinstituicao}} \\
    \MakeUppercase{\organizacaoMD}\\
    \\
    \vspace{0.5cm}
    PORTARIA \textcolor{red}{\portariapublicacao}.
\end{center}

\begin{flushright}

\parbox{6.5cm}{Aprova a Edição da(o) \textcolor{red}{\imprimirtitulo}.}

\end{flushright}

\textbf{O \textcolor{red}{\MakeUppercase{\autoridadeassinante}}}, em conformidade com \textcolor{red}{ALOCAR TODOS OS ÂMPAROS LEGAIS PARA AUTORIDADE ASSINANTE PODER, DE FATO, ASSINAR ESTE DOCUMENTO, JUNTO COM A PORTARIA/DECRETO NO QUAL CONSTA ESSE ÂMPARO}, resolve:

\begin{artenum}[{Art.} 1 {º}]
    \item \textcolor{red}{Primeiro artigo}
    \item \textcolor{red}{Segundo artigo}
    \item \textcolor{red}{N-ésimo artigo}
\end{artenum}

\vspace{3.0cm}

\begin{flushright}

\textcolor{red}{\postoautoridadeassinante} \hspace{} \nomeautoridadeassinante 

\vspace{0.20cm}

\parbox{7.0cm}{\centering \autoridadeassinante} % Tamanho da parbox vai ter que ser ajustado para que o cargo da autoridade assinante fique centralizado com a disposição de seu nome

\end{flushright}

\blfootnote{(Publicação no Boletim do Comando da Aeronáutica \textcolor{red}{\numerobca})}