% Pacotes fundamentais 
\usepackage[brazil]{babel}
\usepackage{cmap}			    % Mapear caracteres especiais no PDF
\usepackage{lmodern}			% Usa a fonte Latin Modern			
\usepackage[T1]{fontenc}		% Selecao de codigos de fonte.
\usepackage[utf8]{inputenc}		% Codificacao do documento (conversão automática dos acentos)
\usepackage{lastpage}			% Usado pela Ficha catalográfica
\usepackage{setspace}
\usepackage{indentfirst}		% Indenta o primeiro parágrafo de cada seção.
\usepackage{color}			    % Controle das cores
\usepackage{graphicx}			% Inclusão de gráficos
\usepackage{pdfpages}
\usepackage{bmpsize}
\usepackage[cmex10]{amsmath}	% Formulas matemáticas
\usepackage{amsfonts,amssymb,latexsym}
\usepackage{mathtools}
\usepackage{bigints}
%\usepackage{amscls}
\usepackage{subfig}
\usepackage[siunitx]{circuitikz}
\usepackage{longtable}
\usepackage{float}
\usepackage{tabularx}
\usepackage{array}
\usepackage{multirow}
\usepackage{pgfplots}
%\pgfplotsset{compat=1.14}
\pgfplotsset{compat=newest}
%\usepackage{geometry} 			% Fazer uma pagina em retrato
\usepackage{pdflscape}
\usepackage{nomencl}
\makenomenclature
\usepackage{textcase}
\usepackage{minibox}
% \usepackage{minipage}

% acrônimos e símbolos usando o pacote glossaries
%\usepackage{hyperref}
%\usepackage[symbols,acronym,nopostdot,nogroupskip,indexonlyfirst]{glossaries}
%\setacronymstyle{short-long}
%\setlength{\glsdescwidth}{0.8\textwidth}
%\setlength{\glspagelistwidth}{0.2\textwidth}
%\makeglossaries

% Pacotes adicionais, usados apenas no âmbito do Modelo Canônico do abnteX2
\usepackage{lipsum}				% para geração de dummy text

% Pacotes de citações
\usepackage[brazilian,hyperpageref]{backref}	% Paginas com as citações na bibl
\usepackage[num, bibjustif]{abntex2cite}		% Citações padrão ABNT: usar alf ou num
\citebrackets[]
\hyphenation{di-mi-nu-i-ção}
\hyphenation{ha-lo-im-plan-ta-dos}

% CONFIGURAÇÕES DE PACOTES
% Configurações do pacote backref
% Usado sem a opção hyperpageref de backref
%\renewcommand{\backrefpagesname}{Citado na(s) página(s):~}
% Texto padrão antes do número das páginas
%\renewcommand{\backref}{}
% Define os textos da citação
%\renewcommand*{\backrefalt}[4]{
%	\ifcase #1 %
%		Nenhuma citação no texto.%
%	\or
%		Citado na página #2.%
%	\else
%		Citado #1 vezes nas páginas #2.%
%	\fi}%

% Configurações de aparência do PDF final
% alterando o aspecto da cor azul
\definecolor{blue}{RGB}{0,0,192}

\pagestyle{plain}

% informações do PDF
\makeatletter
\hypersetup{
     	%pagebackref=true,
		pdftitle={\@title}, 
		pdfauthor={\@author},
    	pdfsubject={\imprimirpreambulo},
	    pdfcreator={LaTeX with abnTeX2},
		pdfkeywords={abnt}{latex}{abntex}{abntex2}{trabalho acadêmico}, 
		colorlinks=true,       		% false: boxed links; true: colored links
    	linkcolor=blue,          	% color of internal links
    	citecolor=blue,        		% color of links to bibliography
    	filecolor=magenta,      	% color of file links
		urlcolor=blue,
		bookmarksdepth=4
}
\makeatother

% Espaçamentos entre linhas e parágrafos 
% Retira espaço extra obsoleto entre as frases.
\frenchspacing 

% O tamanho do parágrafo é dado por:
\setlength{\parindent}{3.0cm}

% Controle do espaçamento entre um parágrafo e outro:
\setlength{\parskip}{0.2cm}  % tente também \onelineskip

% compila o indice
\makeindex

% comandos úteis
\newcommand{\cmmt}[1]{} % comenta trecho de um parágrafo. Ex.: ..texto \cmmt{texto a ser comentado} texto..
\newcommand{\mat}[1]{\mathbf{#1}}
\def\H{^{\mathrm{H}}}
\def\T{^{\mathrm{T}}}

\newcommand\blfootnote[1]{%
  \begingroup
  \renewcommand\thefootnote{}\footnote{#1}%
  \addtocounter{footnote}{-1}%
  \endgroup
}

\usepackage{fancyhdr}            % Permits header customization. See header section below.
% \fancypagestyle{plain}{
%     \lhead{}
%     \fancyhead[R]{\thepage}
%     \fancyhead[L]{}
%     \renewcommand{\headrulewidth}{0pt}
%     \fancyfoot{}
% }

\pagestyle{fancy}
% \fancyhead[R]{\thepage}
% \fancyhead[L]{}
\renewcommand{\headrulewidth}{0pt}
\fancyfoot{}
\renewcommand{\footrulewidth}{0pt}

%\usepackage{showframe}% added to show that the figure is being centered

\usepackage[explicit]{titlesec}
\usepackage[many]{tcolorbox}
\usepackage{lmodern}

\definecolor{titlebgdark}{RGB}{204,204,204}
\definecolor{titlebglight}{RGB}{204,204,204}

\titleformat{name=\chapter}[display]
  {\normalfont\huge\bfseries}
  {}
  {10pt}
  {%
    \begin{tcolorbox}[
      enhanced,
      colback=titlebgdark,
      boxrule=0.25cm,
      colframe=titlebglight,
      arc=0pt,
      outer arc=0pt,
      remember as=title,
      leftrule=0pt,
      rightrule=0pt,
      fontupper=\color{black}\sffamily\bfseries\huge,
      enlarge left by=-1in-\hoffset-\oddsidemargin, 
      enlarge right by=-\paperwidth+1in+\hoffset+\oddsidemargin+\textwidth,
      width=\paperwidth, 
      left=1in+\hoffset+\oddsidemargin, 
      right=\paperwidth-1in-\hoffset-\oddsidemargin-\textwidth,
      top=0.01cm, 
      bottom=0.01cm, 
    ]
    #1
    \end{tcolorbox}%
  }
\titlespacing*{\chapter}
  {-20pt}{-10pt}{0pt}
\makeatother

\PassOptionsToPackage{usenames,dvipsnames,table}{xcolor}
\usepackage{enumerate}
\usepackage[shortlabels]{enumitem} %Referencia para melhorar eventualmente isso: https://ctan.dcc.uchile.cl/macros/latex/contrib/enumitem/enumitem.pdf
\setlistdepth{7}
\newlist{longenum}{enumerate}{7}
% \setlist[longenum]{label=\arabic*, after=\normalfont}
\setlist[longenum,1]{label=\arabic*, after=\normalfont}
\setlist[longenum,2]{label*=.\arabic*, after=\normalfont} 
\setlist[longenum,3]{label*=.\arabic*, after=\normalfont}
\setlist[longenum,4]{label*=.\arabic*, after=\normalfont}
\setlist[longenum,5]{label*=.\arabic*, after=\normalfont}
\setlist[longenum,6]{label*=.\arabic*, after=\normalfont}
\setlist[longenum,7]{label*=.\arabic*, after=\normalfont}

\setlistdepth{3}
\newlist{artenum}{enumerate}{3}
\setlist[artenum,1]{leftmargin=4.5cm}
\setlist[artenum,2]{leftmargin=0.5cm}
\setlist[artenum,3]{leftmargin=0.5cm}


\newcommand\setItemnumber[1]{\setcounter{longenumi}{\numexpr#1-1\relax}}
\newcommand\setItemnumber[1]{\setcounter{enumi}{\numexpr#1-1\relax}}

\renewcommand{\cleardoublepage}{}
\renewcommand{\clearpage}{}

\setlength{\arrayrulewidth}{1pt}
\setlength{\extrarowheight}{1.0pt}

\usepackage{array}
\newcolumntype{P}[1]{>{\centering\arraybackslash}p{#1}}
\newcolumntype{M}[1]{>{\centering\arraybackslash}m{#1}}
\newcolumntype{O}[1]{>{\centering\arraybackslash}X{#1}}
\newcolumntype{K}[1]{>{\centering\let\newline\\\arraybackslash}p{#1}}

\newcommand{\sr}{\rule[-1.5cm]{0pt}{0.9cm}} %Comando para ajustar o tamanho das células devido ao uso de multirow. Defeito: Altura, primeiro elemento, precisa ser setado na mão. Interessante buscar uma forma automatizada que não cause problema como listado abaixo

\newcommand{\adjustheight}[1]{\rule[#1]{0pt}{0.9cm}} %Comando como o mesmo objetivo do comando \sr feito acima, mas podendo passar o argumento para ajustar a altura da célula

% \multirow{N}{\hsize}{\linewidth} : Comando para criar mesclar N células em uma row-wise e deixar o texto alinhado dentro da célula de forma automática. Problema: faz a linha da primeira célula carregar todo o aumento de altura das células mescladas

%%criar um estilo de cabeçalho e rodapé propício documento oficiais da FAB
\makepagestyle{fabstyle}
%%cabeçalhos
\makeevenhead{fabstyle} %%pagina par
 {\thepage/\pageref{LastPage}}
 {}
 {\tipodocumento \hspace{} \numerodocumento}
 
\makeoddhead{fabstyle} %%pagina ímpar ou com oneside
 {\tipodocumento \hspace{} \numerodocumento}
 {}
 {\thepage/\pageref{LastPage}}
\makeheadrule{fabstyle}{\textwidth}{0} %linha


\setlrmarginsandblock{3.5cm}{2.5cm}{*}
\setulmarginsandblock{2.5cm}{*}{1}
\checkandfixthelayout 